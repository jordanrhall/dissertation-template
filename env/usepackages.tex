%%%%%%%%%%%%%%%%%%%%%%%%%%%%%%%%%%%%%%%%%%%%%%%%%%%%%%%%%%%%%%%%%%%%%%%%%%%%%%%%
% usepackages.tex

% ----------------------------------------------------------------------------- %
\usepackage[T1]{fontenc}
\usepackage[utf8]{inputenc}
\usepackage{textcomp}
\usepackage{floatrow}
\usepackage{url}
\usepackage{paralist}
\usepackage{xparse}
\usepackage{ifthen}
\synctex=1
\usepackage{tikz}
\usetikzlibrary{arrows,positioning}
\tikzset{
    %Define standard arrow tip
    >=stealth',
    %Define style for boxes
    punkt/.style={
           rectangle,
           rounded corners,
           draw=black, very thick,
           text width=6.5em,
           minimum height=2em,
           text centered},
    % Define arrow style
    pil/.style={
           ->,
           thick,
           shorten <=2pt,
           shorten >=2pt,}
}


\usepackage{enumerate}
\usepackage[ruled, vlined]{algorithm2e}
\providecommand\algorithmname{algorithm}

% Using ragged2e package to get left-justified, ragged-right text throughout body
\usepackage[document]{ragged2e}
\setlength{\RaggedRightParindent}{0.5in}
\usepackage{etoolbox}
\AtBeginEnvironment{figure}{\setlength{\RaggedRightParindent}{0em}}
\AtBeginEnvironment{table}{\setlength{\RaggedRightParindent}{0em}}
\AtBeginEnvironment{algorithm}{\setlength{\RaggedRightParindent}{0em}}

% sources: https://ctan.org/tex-archive/macros/latex/contrib/algorithm2e/tex

% ----------------------------------------------------------------------------- %
% Float setup
\usepackage{placeins}
\usepackage{ctable}
\floatsetup[table]{framestyle=colorbox,framefit=yes,heightadjust=all,framearound=all,capposition=top}
\floatsetup[figure]{framestyle=colorbox,framefit=yes,heightadjust=all,framearound=all,capposition=bottom}
\usepackage{subcaption}

% ---------------------------------------------------------------------------- %
% Bibtex
\usepackage[authoryear]{natbib}

% ---------------------------------------------------------------------------- %
% Math Stuff
\usepackage{amsmath,amssymb,amsfonts,mathrsfs}

% Image path
\graphicspath{ {figs/}}



\usepackage{mathtools} % for \xmapsto
\usepackage{array}
\renewcommand\arraystretch{0.5}

%%%%%%%%%%%%%%%%%%%%%%%%%%%%%%%%%%%%%%%%%%%%%%%%%%%%%%%%%%%%%%%%%%%%%%%%%%%%%%%%
